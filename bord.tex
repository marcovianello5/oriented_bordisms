% levicivita.tex

% versione preambolo: sabato 11 ottobre 2025

\documentclass[a4paper, 11pt]{amsproc}

% *****************************************************************************
% Il preambolo inizia qui
% *****************************************************************************

% \setcounter{tocdepth}{2}
% \setcounter{chapter}{-1}

% \labelformat{section}{\thechapter.#1}

% \makeatletter
% \def\p@section{\thechapter.}
% \makeatother

\usepackage{preambolo}

% Questo dovrebbe andare da un'altra parte...
\newtcolorbox[auto counter]{richiamo}[2][]{enhanced,float,breakable,title after break={Box~\thetcbcounter: #2 (continuazione)},frame hidden,colbacktitle=black!85!white,fonttitle=\bfseries,title={Box~\thetcbcounter: #2},label={#1}}

% Override di alcuni stili
% \@startsection{name}
% {level}
% {indent}
% {beforeskip}
% {afterskip}
% {style}

\makeatletter
\renewcommand*\paragraph{
  \@startsection{paragraph}%
  {4}%
  {\z@}%
  {.3\linespacing\@plus.7\linespacing}%
  {-\fontdimen2\font}%
  {\normalfont\itshape}%
}
\makeatother

% Pkgs per la gestione dei riferimenti
% È bene che stiano sempre qui alla fine e in questo preciso ordine
\usepackage[italian]{varioref} % per referencing di oggetti "lontani"

\usepackage[
,unicode
,hidelinks % colorlinks || hidelinks
,pdftitle={On the category of bordisms}
,pdfauthor={Marco Vianello}
%,bookmarksdepth=3
]{hyperref}

\usepackage[]{cleveref} % per referencing di oggetti "vicini"
\creflabelformat{enumi}{(#2#1#3)}

% Bourbaki style 
% \usepackage[osf]{Baskervaldx} % tosf in text, tlf in math
% \usepackage[baskervaldx,vvarbb]{newtxmath} % math italic letters from Baskervaldx
% \usepackage[italic,eulergreek]{mathastext}

% *****************************************************************************
% compilazione corrente
% *****************************************************************************

%\includeonly{hamvectf}

% *****************************************************************************
% il documento inizia qui
% *****************************************************************************

\DeclareMathOperator{\tr}{tr}

\begin{document}

\title{Some musings on the category of bordisms}

% \frontmatter

% \maketitle

% {\small Versione: \texttt{\today}}

% % prefazione
% \include{prefazione}

% % toc
% \tableofcontents

% % corpo
% \mainmatter

% \include{capitoli/poly}

\maketitle

I'm writing this set of notes to try pin down to myself the details the underlies the definition of a category of bordisms that is suitable for stating and proving the folklore equivalence between $ 2 $-dimensional topological quantum field theories and Frobenius algebras.

\section{Oriented bordisms}

We start by recalling how the boundary components of an oriented manifold with boundary can be labelled as either ``in'' out ``out''.

\defskip

Let $ X $ be an oriented manifold with boundary. Given a boundary point $ x\in \partial X $, let's say that a vector $ v\in \tb_xX $ is
\begin{enumerate}[(i)]
\item \emph{inward pointing} if there exists a curve $ \gamma\colon \coint{0,\epsilon}\to X $ such that $ \gamma(0) = x $ and $ \dot\gamma(0) = v $;
\item \emph{tangent} to $ \partial X $ if $ v $ belongs to the copy of $ \tb_x(\partial X) $ that sits inside of $ \tb_xX $;
\item \emph{outward-pointing} if there exists a curve $ \gamma\colon \ocint{-\epsilon, 0}\to X $ such that $ \gamma(0) = x $ and $ \dot\gamma(0) = v $.
\end{enumerate}

The following theorem then is true.

\begin{thm}
  Let $ X $ be an oriented manifold with boundary. There always exists a vector field $ v $ along $ \partial X $ such that its value $ v(x) $ is outward-pointing at every point $ x\in \partial X $.
\end{thm}

This result allows us to introduce an orientation on $ \partial X $ as follows. First, take an orientation $ n $-form $ \omega $ for $ X $. Then, consider the interior product $ v\intprod \omega $, an $ (n - 1) $-form on $ X $. Lastly, pull $ v\intprod \omega $ back with the inclusion $ \iota^{\partial X}\colon \partial X\hookrightarrow X $. The resulting $ (n - 1) $-form $ \iota^{\partial X, *}(v\intprod \omega) $ on $ \partial X $ never vanishes, and induces thus an orientation.

\defskip

To see that this orientation is in accordance with the ``Outward Normal First'' rule, let $ \omega $ denote an orientation form for $ X $, and call $ \eta $ the induced orientation form on $ \partial X $. Recall then that a basis $ \{v_1,\dots,v_n\} $ of $ \tb_xX $ is called \emph{positive} whenever
\[
  \scalar{\omega\rvert_x,v_1\wedge\dots \wedge v_n} > 0\text,
\]
and compute the quantity $ \scalar{\eta\rvert_x,v_1\wedge\dots \wedge v_{n - 1}} $ for a basis of $ \{v_1,\dots,v_{n - 1}\} $ of $ \tb_x(\partial X) $.

\defskip

One can distinguish between the ``in'' and ``out'' connected components of the boundary of an oriented manifold as follows. Given $ x\in \partial X $, call a vector $ w\in \tb_xX $ a \emph{positive normal} for $ \partial X $ at $ x $ if\footnote{Sloppily, but I hope it's clear what I mean. Here ``out top'' mean ``as the last one''.} adding it on top of a positive basis of $ \tb_x(\partial X) $ results in a positive basis of $ \tb_xX $. The following theorem holds.

\begin{thm}
  Let $ X $ be a manifold with boundary. Let $ \Sigma $ be a connected component of $ \partial X $. The inward/outward direction of a positive normal vector at a point of $ \Sigma $ does not depend on the particular positive normal vector chosen, nor on the particular point.
\end{thm}

This theorem allows us to separate the boundary components whose positive normals are inward-pointing from those whose positive normals are outward-pointing. The former are called \emph{in-bondaries}, the latter \emph{out-boundaries} of $ X $.

\defskip

We're now ready to spell out the definition of oriented bordism.

\begin{dfn}[Oriented bordisms]
  Let $ \Sigma_0 $ and $ \Sigma_1 $ be two oriented closed $ (n - 1) $-dimensional manifolds. A \emph{bordism} between $ \Sigma_0 $ and $ \Sigma_1 $ is a triple $ (B, \theta_0, \theta_1) $ where
  \begin{itemize}
  \item $ B $ is a compact $ n $-dimensional manifold with boundary;
  \item $ \theta_0 $ and $ \theta_1 $ are orientation-preserving embeddings
    \[
      \theta_0\colon \Sigma_0\times \coint{0,\epsilon}\to M\qquad \theta_1\colon \Sigma_1\times \ocint{1 - \epsilon,1}\to M
    \]
    such that $ \theta_0(x,0) $ belongs to an in-boundary of $ B $ for every $ x\in \Sigma_0 $ and $ \theta_1(x,1) $ belongs to an out-boundary of $ B $ for every $ x\in \Sigma_1 $, and such that the restrictions
    \[
      \Sigma_0\times \{0\}\to \coprod\{\text{in-boundaries of $ B $}\}\qquad \Sigma_1\times \{1\}\to \coprod\{\text{out-boundaries of $ B $}\}
    \]
    are (orientation-preserving) diffeomorphisms.
  \end{itemize}
\end{dfn}

We write $ (B,\theta_0,\theta)\colon \Sigma_0\to \Sigma_1 $ or even $ B\colon \Sigma_0\to \Sigma_1 $ to denote that $ (B,\theta_0,\theta_1) $ is a bordism from $ \Sigma_0 $ to $ \Sigma_1 $.

\defskip

Before going on let's recall the notion of collar neighbourhood.

\begin{dfn}[Collars]
  Let $ X $ be a manifold with boundary. Let $ \theta\colon \partial X\times \coint{0,\epsilon}\to X $ be an embedding such that
  \[
    \begin{tikzcd}[column sep=large]
      \partial X\ar[dr, hook]\ar[r, "{x\mapsto (x,0)}"] & \partial X\times \coint{0,\epsilon}\ar[d, "\theta"]\\
      {} & X
    \end{tikzcd}
  \]
  commutes. Then we say that $ \theta $ defines a \emph{collar neighbourhood} of $ \partial X $.
\end{dfn}

Keep in mind that, despite their name, to specify a collar neighbourhood the entire embedding $ \partial X\times \coint{0,\epsilon}\to X $ is needed, not only its image inside of $ X $.

\defskip

The following theorem is worth mentioning. It will be useful later when we will ``glue'' bordism together.

\begin{thm}
  \todo[inline]{diocane}
\end{thm}

\begin{ex}[Constructing bordisms]
  Let $ S^1 $ denote the (counterclockwise-)oriented $ 1 $-dimensional sphere, and let's flip its orientation in $ \bar S^1 $. Take the oriented manifold with boundary $ B = S^1\times \ccint{0,1} $. Denote with $ \emptyset^2 $ the $ 2 $-dimensional null manifold. Write all the $ 2 $-dimensional bordisms 
  \begin{multicols}{2}
    \begin{itemize}
    \item $ \emptyset^2\to S^1 $;
    \item $ S^1\to S^1 $;
    \item $ S^1\to S^1 $;
    \item $ \bar S^1\to S^1 $;
    \item $ S^1\to \bar S^1 $;
    \item $ \bar S^1\to \bar S^1 $;
    \item $ S^1\sqcup S^1\to \emptyset^2 $;
    \item $ S^1\sqcup \bar S^1\to \emptyset^2 $;
    \item $ \bar S^1\sqcup S^1\to \emptyset^2 $;
    \item $ \emptyset^2\to S^1\sqcup S^1 $;
    \item $ \emptyset^2\to S^1\sqcup \bar S^1 $;
    \item $ \emptyset^2\to \bar S^1\sqcup S^1 $;
    \end{itemize}
  \end{multicols}
  that have $ B $ as bordant manifold.
\end{ex}

% bibliografia
% \backmatter
\printbibliography[title=Libri] % headig=bibintoc

\end{document}

%%% Local Variables:
%%% TeX-master: t
%%% End:
